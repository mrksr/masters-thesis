\usepackage{tikz}
\usepackage{pgfplots}
\pgfplotsset{compat=1.12}

\newcommand{\plotdatapath}{\figurepath/../../thesis/figures/data}
\newcommand{\includestandalonewithpath}[2][]{%
    \begingroup%
    \StrCount{#2}{/}[\matches]%
    \StrBefore[\matches]{#2}{/}[\figurepath]%
    \includestandalone[#1]{#2}%
    \endgroup%
}

% See
% https://tex.stackexchange.com/a/23583
\tikzset{onslide/.code args={<#1>#2}{%
  \only<#1>{\pgfkeysalso{#2}} % \pgfkeysalso doesn't change the path
}}

\usetikzlibrary{angles}
\usetikzlibrary{calc}
\usetikzlibrary{decorations}
\usetikzlibrary{positioning}
\usetikzlibrary{quotes}
\usetikzlibrary{shapes}
\usetikzlibrary{shapes.multipart}

\usepgfplotslibrary{fillbetween}
\usepgfplotslibrary{units}

\tikzstyle{very small node} = [draw, circle, fill=tumblue!20, thick, minimum size=3pt, inner sep=0pt]
\tikzstyle{data point} = [very small node]
\tikzstyle{small node} = [draw, circle, fill=tumblue!20, thick, minimum size=6pt, inner sep=0pt]
\tikzstyle{large node} = [draw, circle, fill=tumblue!20, thick, minimum size=5pt, inner sep=5pt]

\tikzstyle{edge} = [draw, thick, shorten >=2pt, shorten <=2pt]
\tikzstyle{directed} = [arrows={-latex}, shorten >=2pt, shorten <=0pt]
\tikzstyle{inverse directed} = [arrows={latex-}, shorten >=0pt, shorten <=2pt]
\tikzstyle{no shorten} = [shorten >=0pt, shorten <=0pt]
\tikzstyle{marked} = [tumred]
\tikzstyle{marked edge} = [very thick, marked]
\tikzstyle{marked node} = [very thick, draw=tumred, fill=tumred!20]
\tikzstyle{every text node part} = [align=center, execute at begin node=\setlength{\baselineskip}{2ex}]

\pgfplotsset{unit code/.code 2 args={\si{#1#2}}}
\pgfplotsset{default plot/.style = {
    grid=major,
    minor y tick num=1,
    minor x tick num=1,
    width=200pt,
    height=172pt,
}}
\pgfplotsset{xy plot/.style = {
    default plot,
    width=220pt,
    height=220pt,
    xmin=-26,
    xmax=26,
    ymin=-26,
    ymax=26,
    xlabel={$x$ position},
    x unit={\meter},
    ylabel={$y$ position},
    y unit={\meter},
}}

\pgfplotsset{xy trajectory/.style = {
    ultra thick,
    smooth,
}}
\tikzstyle{success} = [tumgreen]
\tikzstyle{success node} = [fill=tumgreen!15]
\tikzstyle{fail} = [tumred]
\tikzstyle{fail node} = [fill=tumred!15, rectangle]
\newcommand{\xytrajectory}[2]{
    \addplot[xy trajectory, #2] table[
        ignore chars={\#}, col sep=space,
        x expr={(\thisrow{next_r_zero}) * cos(\thisrow{next_psi_zero} r)},
        y expr={(\thisrow{next_r_zero}) * sin(\thisrow{next_psi_zero} r)},
    ] {\plotdatapath/solution_xy_trajectories_#1.xa}
    % Starting state
    node[pos=0, very small node, #2 node] {}
    % Hacky way to make every trajectory just a little bit longer
    node[pos=1, sloped, fill, inner sep=0, minimum width=1.6pt, minimum height=1.6pt] {};
}
\newcommand{\goalarea}{
    \draw[draw=tumblue, fill=tumblue!15] \pgfextra{
        \pgfpathellipse{\pgfplotspointaxisxy{0}{0}}
        {\pgfplotspointaxisdirectionxy{5}{0}}
        {\pgfplotspointaxisdirectionxy{0}{5}}
    };
    \node[tumblue, font=\footnotesize] at (0, 0) {Goal};
}

\pgfplotsset{with error/.style = {
        error bars/.cd, y dir=both, y explicit,
}}
\pgfplotsset{simulation/.style = {
    black, dashed, thick
}}
\pgfplotsset{means only/.style = {
    tumviolet
}}
\pgfplotsset{means only fill/.style = {
    fill=tumviolet!15
}}
\pgfplotsset{one step/.style = {
    tumgreen
}}
\pgfplotsset{one step fill/.style = {
    fill=tumgreen!15
}}
\pgfplotsset{linearization/.style = {
    tumblue
}}
\pgfplotsset{linearization fill/.style = {
    fill=tumblue!15
}}
\newcommand{\goalpercentageline}[4]{
    \addplot[
        thick, #1,
        mark=#4, mark size=1.5pt,
        only marks,
        with error,
        ] table[
        ignore chars={\#}, col sep=comma,
        x=M, y=#2_percentage, y error=#2_percentage_sem
    ] {\plotdatapath/results_goal_percentage.dat};
    \addplot[forget plot, thin, opacity=0, name path=#1a] table[ignore chars={\#}, col sep=comma, x=M,
        y expr={\thisrow{#2_percentage} + \thisrow{#2_percentage_sem}}
    ] {\plotdatapath/results_goal_percentage.dat};
    \addplot[forget plot, thin, opacity=0, name path=#1b] table[ignore chars={\#}, col sep=comma, x=M,
        y expr={\thisrow{#2_percentage} - \thisrow{#2_percentage_sem}}
    ] {\plotdatapath/results_goal_percentage.dat};
    \addplot[forget plot, #1, opacity=0.25] fill between[of=#1a and #1b, on layer={axis background}];
    \addlegendentry{#3}
}

% See
% https://tex.stackexchange.com/a/21759/34265
\tikzset{
    right angle quadrant/.code={
        \pgfmathsetmacro\quadranta{{1,1,-1,-1}[#1-1]}     % Arrays for selecting quadrant
        \pgfmathsetmacro\quadrantb{{1,-1,-1,1}[#1-1]}},
    right angle quadrant=1, % Make sure it is set, even if not called explicitly
    right angle length/.code={\def\rightanglelength{#1}},   % Length of symbol
    right angle length=2ex, % Make sure it is set...
    right angle symbol/.style n args={3}{
        insert path={
            let \p0 = ($(#1)!(#3)!(#2)$) in     % Intersection
                let \p1 = ($(\p0)!\quadranta*\rightanglelength!(#3)$), % Point on base line
                \p2 = ($(\p0)!\quadrantb*\rightanglelength!(#2)$) in % Point on perpendicular line
                let \p3 = ($(\p1)+(\p2)-(\p0)$) in  % Corner point of symbol
            (\p1) -- (\p3) -- (\p2)
        }
    }
}
