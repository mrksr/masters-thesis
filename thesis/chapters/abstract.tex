\begin{Abstract}{english}
In reinforcement learning, an agent has to learn how to make decisions in an unknown environment in order to maximize a numerical reward.
In model-based reinforcement learning, the experience gained via interaction is represented as a transition model which can be used to simulate the system's future behaviour.
This thesis is concerned with reducing the model bias introduced by choosing actions which are optimal with respect to an imperfect model.
Instead of relying on a single deterministic model, gathered knowledge is represented using Gaussian processes which encode a probability distribution over all plausible transition models.
By averaging over all of them, the expected long-term reward is calculated, which explicitly incorporates model uncertainties into long-term planning.
A controller is formulated by applying particle swarm optimization to this expected reward, directly choosing appropriate actions.
Besides formally introducing these tools, this thesis investigates their effectiveness on a benchmark problem with the task of learning how to balance and navigate a bicycle.
Thereby, multiple approaches of incorporating uncertainties are described and compared to the classic technique of deterministic predictions.
\end{Abstract}

\begin{Abstract}{ngerman}
Im Reinforcement Learning ist es die Aufgabe eines Agenten zu lernen, welche Entscheidungen in einer unbekannten Umgebung eine numerische Belohnung maximieren.
Im modellbasierten Reinforcement Learning wird die durch Interaktion erworbene Erfahrung in einem Transitionsmodell repräsentiert, mit dem das zukünftige Verhalten des Systems simuliert werden kann.
Diese Arbeit beschäftigt sich damit, den systematischen Fehler zu reduzieren, der durch Entscheidungsfindung auf Basis eines imperfekten Modells entsteht.
Anstatt eines einzelnen deterministischen Modells wird Wissen mit Hilfe von Gaußprozessen dargestellt, die eine Wahrscheinlichkeitsverteilung über alle plausiblen Transitionsmodelle darstellen.
Unter Berücksichtigung aller dieser Modelle wird die erwartete Belohnung über mehrere Zeitschritte errechnet, wodurch Modellunsicherheiten explizit in die Planung integriert werden.
Durch Anwendung von Partikelschwarmoptimierung auf dem erwarteten Reward wird eine Entscheidungsstrategie formuliert, die Aktionen direkt mit Hilfe der Modelle wählt.
Neben der formalen Beschreibung dieser Werkzeuge untersucht diese Arbeit ihre Effektivität anhand eines Beispielproblems mit der Aufgabe, ein Fahrrad zu balancieren und zu navigieren.
Dabei werden mehrere Ansätze beschrieben, wie Unsicherheiten in die Planung integriert werden können, und mit dem klassischen Ansatz deterministischer Vorhersagen verglichen.
\end{Abstract}
