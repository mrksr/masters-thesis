\thispagestyle{scrplain}
\begin{abstractFrame}

\begin{enAbstract}
This thesis deals with the construction of optimized flight paths for tracking shots using multicopters.
The theory chapter defines appropriate metrics which are then used in an optimization problem to create spline trajectories.
This solution is implemented in a software suite capable of simulating flights.
Finally, after creating an interface to UAVs by \asctec, the paths are tested in an exemplary use case.
\end{enAbstract}

\begin{deAbstract}
Diese Arbeit beschäftigt sich mit der Konstruktion von optimierten Flugpfaden für Kamerafahrten mit Multikoptern.
Im Theorieteil werden geeignete Metriken definiert, mittels derer in einem Optimierungsverfahren Splines als Trajektorien erzeugt werden.
Diese Lösung wird in eine Software überführt und um eine Flugsimulation erweitert.
Schließlich werden Schnittstellen zu Flugsystemen der Firma \asctec geschaffen und die Flugbahnen in einem Anwendungsfall getestet.
\end{deAbstract}

\begin{svAbstract}
This thesis deals with the construction of optimized flight paths for tracking shots using multicopters.
The theory chapter defines appropriate metrics which are then used in an optimization problem to create spline trajectories.
This solution is implemented in a software suite capable of simulating flights.
Finally, after creating an interface to UAVs by \asctec, the paths are tested in an exemplary use case.
\end{svAbstract}

\end{abstractFrame}
