\usepackage{tikz}
\usepackage{pgfplots}
\pgfplotsset{compat=1.12}

\usetikzlibrary{angles}
\usetikzlibrary{calc}
\usetikzlibrary{positioning}
\usetikzlibrary{quotes}
\usetikzlibrary{shapes.multipart}

\tikzstyle{very small node} = [draw, circle, fill=tumblue!15, thick, minimum size=3pt, inner sep=0pt]
\tikzstyle{small node} = [draw, circle, fill=tumblue!15, thick, minimum size=6pt, inner sep=0pt]
\tikzstyle{large node} = [draw, circle, fill=tumblue!15, thick, minimum size=5pt, inner sep=5pt]

\tikzstyle{edge} = [draw, thick, shorten >=2pt, shorten <=2pt]
\tikzstyle{directed} = [arrows={-latex}, shorten >=2pt, shorten <=0pt]
\tikzstyle{marked} = [tumred]
\tikzstyle{marked edge} = [very thick, marked]
\tikzstyle{marked node} = [very thick, draw=tumred, fill=tumred!15]
\tikzstyle{every text node part} = [align=center, execute at begin node=\setlength{\baselineskip}{2ex}]

% See
% https://tex.stackexchange.com/a/21759/34265
\tikzset{
    right angle quadrant/.code={
        \pgfmathsetmacro\quadranta{{1,1,-1,-1}[#1-1]}     % Arrays for selecting quadrant
        \pgfmathsetmacro\quadrantb{{1,-1,-1,1}[#1-1]}},
    right angle quadrant=1, % Make sure it is set, even if not called explicitly
    right angle length/.code={\def\rightanglelength{#1}},   % Length of symbol
    right angle length=2ex, % Make sure it is set...
    right angle symbol/.style n args={3}{
        insert path={
            let \p0 = ($(#1)!(#3)!(#2)$) in     % Intersection
                let \p1 = ($(\p0)!\quadranta*\rightanglelength!(#3)$), % Point on base line
                \p2 = ($(\p0)!\quadrantb*\rightanglelength!(#2)$) in % Point on perpendicular line
                let \p3 = ($(\p1)+(\p2)-(\p0)$) in  % Corner point of symbol
            (\p1) -- (\p3) -- (\p2)
        }
    }
}
