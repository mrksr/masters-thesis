\documentclass[
    compress,
    10pt,
    c,
    aspectratio=1610
]{beamer}
\usefonttheme{professionalfonts}

%%%%%%%%%%%%%%%%
%  Standalone  %
%%%%%%%%%%%%%%%%
% We load currfile explicitly and very early to avoid empty stack warnings.
\usepackage{currfile}
\usepackage{standalone}
\usepackage{xstring} % standalone path magic

%%%%%%%%%%%%%%
%  Language  %
%%%%%%%%%%%%%%
\usepackage[english]{babel}
\uselanguage{English}
\languagepath{English}

%%%%%%%%%%%
%  Fonts  %
%%%%%%%%%%%
\usepackage[T1]{fontenc}
\usepackage[utf8]{inputenc}

% You might want to change this
% See http://www.tug.dk/FontCatalogue
\usepackage{arev}
\usepackage{arevmath}
\usepackage{microtype}

%%%%%%%%%%%%%%%%%%%%%%%%%%%%%
%  Used by beamer or theme  %
%%%%%%%%%%%%%%%%%%%%%%%%%%%%%
\usepackage{url}
\usepackage{listings}
\usepackage{xcolor}
\usepackage{xspace}

%%%%%%%%%%
%  Math  %
%%%%%%%%%%
\usepackage{amssymb}
\usepackage{amsfonts}
\usepackage{amsmath}
\usepackage{mathtools}
\mathtoolsset{showonlyrefs,showmanualtags}
\usepackage{xfrac}
\usepackage{bm}
\usepackage{calc} % widthof

%%%%%%%%%%%%
%  Tables  %
%%%%%%%%%%%%
\usepackage{tabularx}
\usepackage{booktabs}

%%%%%%%%%%
%  Misc  %
%%%%%%%%%%
\usepackage{csquotes}
\usepackage{siunitx}

%%%%%%%%%%%%%%%%%%
%  Beamer Theme  %
%%%%%%%%%%%%%%%%%%
\graphicspath{{.}{./figures/}{./preamble/}}
\usetheme{MA}
\setbeamercovered{invisible}
% Add slide-circles without subsections
\AtBeginSection[]{\subsection{}}

%%%%%%%%%%%%%%%%%%%%%%%%%%%%%%%%%%%%%%%%%%%%%%%%%%%%%%%%%%%%%%%%%%%%%%%
%                               Colors                                %
%%%%%%%%%%%%%%%%%%%%%%%%%%%%%%%%%%%%%%%%%%%%%%%%%%%%%%%%%%%%%%%%%%%%%%%
% Base Colors Corporate Design
\definecolor{tumblue}{HTML}{0065BD}
\definecolor{tumgreen}{HTML}{A2AD00}
\definecolor{tumorange}{HTML}{E37222}
\definecolor{tumivory}{HTML}{DAD7CB}
\definecolor{tumred}{HTML}{E53418} % not in Styleguide

% Derived Colors
% https://kuler.adobe.com/create/color-wheel/
% https://portal.mytum.de/corporatedesign/print/styleguide/
% Grays - TUM
\definecolor{tumgray0}{HTML}{000000}
\definecolor{tumgray1}{HTML}{58585A}
\definecolor{tumgray2}{HTML}{9C9D9F}
\definecolor{tumgray3}{HTML}{D9DADB}
\definecolor{tumgray4}{HTML}{FFFFFF}

% Blues - TUM
\definecolor{tumblue0}{HTML}{003359}
\definecolor{tumblue1}{HTML}{005293}
\definecolor{tumblue2}{HTML}{0073CF}
\definecolor{tumblue3}{HTML}{64A0C8}
\definecolor{tumblue4}{HTML}{98C6EA}

% Greens - Adobe
\definecolor{tumgreen0}{HTML}{EAF900}
\definecolor{tumgreen1}{HTML}{AEBA00}
\definecolor{tumgreen2}{HTML}{8A9300}
\definecolor{tumgreen3}{HTML}{525800}

% Reds - Adobe
\definecolor{tumred0}{HTML}{F23719}
\definecolor{tumred1}{HTML}{CB2E15}
\definecolor{tumred2}{HTML}{90210F}
\definecolor{tumred3}{HTML}{65170B}

% Oranges - Adobe
\definecolor{tumorange0}{HTML}{F07824}
\definecolor{tumorange1}{HTML}{C9651E}
\definecolor{tumorange2}{HTML}{8E4715}
\definecolor{tumorange3}{HTML}{63320F}

% Color Aliases
\definecolor{clCommon}{named}{tumblue0}

\definecolor{layoutTitle}{named}{tumblue}
\definecolor{layoutChap}{named}{clCommon}
\definecolor{layoutSec}{named}{clCommon}
\definecolor{layoutSubSec}{named}{clCommon}
\definecolor{layoutSubSubSec}{named}{clCommon}
\definecolor{layoutParagraph}{named}{clCommon}

%%%%%%%%%%%%%%%%%%%%%%%%%%%%%%%%%%%%%%%%%%%%%%%%%%%%%%%%%%%%%%%%%%%%%%%
%                         Frames and Theorems                         %
%%%%%%%%%%%%%%%%%%%%%%%%%%%%%%%%%%%%%%%%%%%%%%%%%%%%%%%%%%%%%%%%%%%%%%%
%%%%%%%%%%%
%  Boxes  %
%%%%%%%%%%%
\newenvironment{defaultColourFrame}[2]
{
\vspace*{\parskip}
    \begin{mdframed}[innerleftmargin=4pt, innerrightmargin=4pt, innertopmargin=4pt, innerbottommargin=4pt, linecolor=#1, middlelinewidth=0.5pt, backgroundcolor=#2, splitbottomskip=4pt, splittopskip=12pt, innermargin=4pt]
}
{
    \end{mdframed}
}

%%%%%%%%%%%%%%%%%%%
%  Code Listings  %
%%%%%%%%%%%%%%%%%%%
% Allow local minted styles
\usemintedstyle{tum}
\newminted{c}{bgcolor=tumgreen!20, linenos}
\newminted{yaml}{bgcolor=tumgreen!20, linenos}

%%%%%%%%%%%%%%
%  Theorems  %
%%%%%%%%%%%%%%
\theoremstyle{break}
\theoremheaderfont{\rmfamily\bfseries\boldmath}
\theorembodyfont{\normalfont}
\theoremseparator{}
\theorempreskipamount 0pt
\theorempostskipamount 0pt
%\theoremprework{\vspace*{-\parskip}}

\newtheorem{DefBox}[Definition]{Definition}
\newtheorem{TheoBox}[Definition]{Satz}
\newtheorem{LemBox}[Definition]{Lemma}
\newtheorem{ProbBox}[Definition]{Problem}

\renewenvironment{definition}
{
    \begin{defaultColourFrame}{tumblue!15}{tumblue!15}
    \begin{DefBox}
}
{
    \end{DefBox}
    \end{defaultColourFrame}
}
\renewenvironment{Definition}{\begin{definition}}{\end{definition}}

\newenvironment{problem}
{
    \begin{defaultColourFrame}{tumgreen!15}{tumgreen!15}
    \begin{ProbBox}
}
{
    \end{ProbBox}
    \end{defaultColourFrame}
}
\newenvironment{Problem}{\begin{problem}}{\end{problem}}

\renewenvironment{theorem}
{
    \begin{defaultColourFrame}{tumgreen!15}{tumgreen!15}
    \begin{TheoBox}
}
{
    \end{TheoBox}
    \end{defaultColourFrame}
}
\renewenvironment{Theorem}{\begin{theorem}}{\end{theorem}}

\renewenvironment{lemma}%
{
    \begin{defaultColourFrame}{tumgreen!15}{tumgreen!15}
    \begin{LemBox}
}
{
    \end{LemBox}
    \end{defaultColourFrame}
}
\renewenvironment{Lemma}{\begin{lemma}}{\end{lemma}}

\makeatletter
\renewenvironment{proof}
{\textbf{\proofname}\par\nobreak\@afterheading\vspace*{-\parskip}}
{\qed}

\renewenvironment{Proof}
{\textbf{\proofname}\par\nobreak\@afterheading\vspace*{-\parskip}}
{\qed}
\makeatother

\crefname{Algorithm}{algorithm}{algorithms}
\Crefname{Algorithm}{Algorithm}{Algorithms}

%%%%%%%%%%%%%%%%%%%%%%%%%%%%%%%%%%%%%%%%%%%%%%%%%%%%%%%%%%%%%%%%%%%%%%%
%                              Pictures                               %
%%%%%%%%%%%%%%%%%%%%%%%%%%%%%%%%%%%%%%%%%%%%%%%%%%%%%%%%%%%%%%%%%%%%%%%
\graphicspath{{figures/}}

%%%%%%%%%%%%%%%%%%%%%%%%%%%%%%%%%%%%%%%%%%%%%%%%%%%%%%%%%%%%%%%%%%%%%%%
%                           Layout Elements                           %
%%%%%%%%%%%%%%%%%%%%%%%%%%%%%%%%%%%%%%%%%%%%%%%%%%%%%%%%%%%%%%%%%%%%%%%
%%%%%%%%%%%%%%
%  Sections  %
%%%%%%%%%%%%%%
\addtokomafont{chapter}{\color{layoutChap}}
\addtokomafont{section}{\color{layoutSec}}
\addtokomafont{subsection}{\color{layoutSubSec}}
\addtokomafont{subsubsection}{\color{layoutSubSubSec}}
\addtokomafont{paragraph}{\color{layoutParagraph}}
\addtokomafont{disposition}{\boldmath}

%%%%%%%%%%%%%%
%  Abstract  %
%%%%%%%%%%%%%%
\newenvironment{myAbstract}[1]
{
    \begin{otherlanguage}{#1}
        \chapter*{\abstractname}
}
{
    \end{otherlanguage}
}

%%%%%%%%%%
%  Math  %
%%%%%%%%%%
\DeclareRobustCommand{\qed}{%
  \ifmmode \square%\mathqed
  \else
    \leavevmode\unskip\penalty9999 \hbox{}\nobreak\hfill
    \quad\hbox{$\square$}%
  \fi
}
% Fix some greek letters
% See http://tex.stackexchange.com/a/33141
% And use texdef -t latex -p mathpazo varrho to find definitions from which to
% subtract 7000.
\renewcommand{\pi}{\mathchar"119}
\renewcommand{\rho}{\mathchar"11A}
\renewcommand{\varrho}{\mathchar"125}

\sisetup{per-mode=symbol}

%%%%%%%%%%%%
%  Widths  %
%%%%%%%%%%%%
\newcommand{\tablewidth}{.975\textwidth}
\newcommand{\figurewidth}{.975\textwidth}
\newcommand{\subfigurewidth}{.485\textwidth}

\makeatletter

\newcommand\thesisTypeText{}
\newcommand\thesisType[1]{\renewcommand\thesisTypeText{#1}}

\newcommand\matNoText{}
\newcommand\matNo[1]{\renewcommand\matNoText{#1}}

\newcommand\studiesText{}
\newcommand\studies[1]{\renewcommand\studiesText{#1}}

\newcommand\eMailText{}
\newcommand\eMail[1]{\renewcommand\eMailText{#1}}

\newcommand\universityText{}
\newcommand\university[1]{\renewcommand\universityText{#1}}

\newcommand\chairText{}
\newcommand\chair[1]{\renewcommand\chairText{#1}}

\newcommand\advisorText{}
\newcommand\advisor[1]{\renewcommand\advisorText{#1}}

\newcommand\tutorText{}
\newcommand\tutor[1]{\renewcommand\tutorText{#1}}

\renewcommand\maketitle{

    \begin{titlepage}%

        \let\footnotesize\small
        \let\footnoterule\relax
        \let \footnote \thanks

        %\begin{center}
        %  \includegraphics[width=0.75\linewidth,keepaspectratio]{bilder/unilogo.png}
        %\end{center}

        %  \hspace{-.1\linewidth}
        %  \parbox{1.2\linewidth}
        %  {
        %    \raggedleft
        %    \Large \textbf{\@author}
        %
        %    \small
        %    \studiesText \ (\matNoText) \\
        %    \eMailText
        %  }
        \hspace*{-.1\linewidth}
        \parbox{1.2\linewidth}
        {
            \raggedleft
            \small \null%\@date
        }
        \vspace*{\fill}
        %\vskip 2em%

        \hspace*{-.1\linewidth}
        \parbox{1.2\linewidth}
        {
            \raggedleft

            %\begin{tabular*}{\linewidth}[b]{@{}p{0.5\linewidth}@{}p{0.5\linewidth}@{}}
            % \@date  &  \raggedleft  \thesisTypeText
            %\end{tabular*}
            \thesisTypeText \\ %-- \@date \\
            {\color{layoutTitle}\rule[.9em]{\linewidth}{2pt}
                \huge {\bfseries\rmfamily\boldmath \@title}

                \rule{\linewidth}{2pt}\par}

            \vskip 0.7em%

            \Large {\bfseries\rmfamily\@author}

            %\small
            %\studiesText \ (\matNoText) \\
            %\eMailText

        }%

        \vspace*{\fill}

        %\hspace*{-.1\linewidth}
        %\parbox{1.2\linewidth}
        %{
                    %\raggedleft
                    %\Large
                    %\textrm{\rmfamily\advisorText} \\
                    %\small
                    %Gutachter

                    %\vskip 1.5em

                    %\raggedleft
                    %\Large
                    %\textrm{\rmfamily\tutorText} \\
                    %\small
                    %Betreuer
        %}

        %\vfil\null
        %\vspace*{\fill}

        \hspace*{-.1\linewidth}
        \parbox{1.2\linewidth}
        {

            \begin{tabular}[c]{@{}l@{}r@{}}
                \parbox[b][][b]{0.35\linewidth}
                {
                    \includegraphics[width=\linewidth,keepaspectratio]{figures/tum.eps}
                }

                &

                \parbox[b][][b]{0.65\linewidth}
                {
                    \raggedleft
                    \textbf{\rmfamily\chairText} \\
                    \small
                    \universityText
                }

            \end{tabular}
        }

        %\vskip 2.5em%

        %{\large \@date \par}%       % Set date in \large size.

        %\vfil\null

    \end{titlepage}%
    \setcounter{footnote}{0}%
    %\global\let\thanks\relax
    %\global\let\maketitle\relax
    %\global\let\@thanks\@empty
    %\global\let\@author\@empty
    %\global\let\@date\@empty
    %\global\let\@title\@empty
    %\global\let\title\relax
    %\global\let\author\relax
    %\global\let\date\relax
    %\global\let\and\relax
}
\makeatother

% Sets
\newcommand{\N}{\mathbb{N}}
\newcommand{\Z}{\mathbb{Z}}
\newcommand{\R}{\mathbb{R}}
\newcommand{\Es}{\mathcal{S}}
\newcommand{\Ah}{\mathcal{A}}
\newcommand{\Tee}{\mathcal{T}}
\newcommand{\D}{\mathcal{D}}

% Random Variable
\newcommand{\rv}[1]{\bm{#1}}
% Matrix
\newcommand{\mat}[1]{\bm{#1}}
\newcommand{\Eye}{\mathrm{I}}
% Pseudo Inputs
\newcommand{\ps}[1]{\bar{#1}}
\newcommand{\psmat}[1]{\ps{\mat{#1}}}

% Nicer empty set
\renewcommand{\emptyset}{\varnothing}

% Math operators
% General
\DeclareMathOperator*{\argmax}{argmax}
\DeclareMathOperator*{\argmin}{argmin}
\DeclareMathOperator{\sgn}{sgn}
\DeclareMathOperator{\diag}{diag}
\DeclareMathOperator*{\maximize}{maximize}
\DeclareMathOperator*{\minimize}{minimize}
\DeclareMathOperator{\subjectto}{subject\ to}
\DeclareMathOperator{\Oh}{\mathcal{O}}
\newcommand{\Powerset}[1]{2^{#1}}
\newcommand*{\diff}{\mathop{}\!\mathrm{d}}
\DeclarePairedDelimiter{\abs}{\vert}{\vert}

% Bicycle
\DeclareMathOperator{\DynBicycle}{\Dyn_{\text{bicycle}}}
\DeclareMathOperator{\RwdBicycle}{\Rwd_{\text{bicycle}}}
\DeclareMathOperator{\Goal}{\Tee_{\text{goal}}}
\DeclareMathOperator{\Fallen}{\Tee_{\text{fallen}}}

% RL
\DeclareMathOperator{\Dists}{\mathcal{P}}
\DeclareMathOperator{\Rwd}{\mathbf{R}}
\DeclareMathOperator{\Vlu}{\mathbf{V}}
\DeclareMathOperator{\Dyn}{\mathbf{f}}
\DeclareMathOperator{\AVlu}{\hat{\Vlu}}

% PSO
\DeclareMathOperator{\Pe}{\mathcal{P}}
\DeclareMathOperator{\Pos}{\mathbf{x}}
\DeclareMathOperator{\Vel}{\mathbf{v}}
\DeclareMathOperator{\Neigh}{\mathcal{N}}
\DeclareMathOperator{\Cog}{\mathbf{y}}
\DeclareMathOperator{\Soc}{\hat{\Cog}}

% Probabilities
\DeclareMathOperator{\E}{\mathbb{E}}
\DeclareMathOperator{\cov}{cov}
\DeclareMathOperator{\var}{var}
\DeclareMathOperator{\corr}{\varrho}
\DeclareMathOperator{\p}{p}
\DeclareMathOperator{\K}{\mathcal{K}}
\DeclareMathOperator{\Q}{\mathcal{Q}}
\DeclareMathOperator{\Norm}{\mathcal{N}}
\DeclareMathOperator{\Uni}{\mathbb{U}}
\DeclareMathOperator{\GP}{\mathcal{GP}}
\DeclareMathOperator{\SPGP}{\mathcal{SPGP}}
\DeclareMathOperator{\Ell}{\mathcal{L}}
\DeclareMathOperator{\X}{\mathcal{X}}

\providecommand\given{}
\DeclarePairedDelimiterX{\Cond}[1]{(}{)}{
\renewcommand\given{%
  \nonscript\:
  \delimsize\vert
  \nonscript\:
  \mathopen{}
  \allowbreak}
#1
}
\newcommand{\Prob}{\p\Cond}
\newcommand{\Gaussian}{\Norm\Cond}
\newcommand{\Uniform}{\Uni\Cond}

\DeclarePairedDelimiterXPP{\Moment}[2]{#1}{[}{]}{}{
\renewcommand\given{%
  \nonscript\:
  \delimsize\vert
  \nonscript\:
  \mathopen{}
  \allowbreak}
#2
}

\providecommand\with{}
\DeclarePairedDelimiterX{\Set}[1]{\{}{\}}{
\renewcommand\with{%
  \nonscript\:
  \delimsize\vert
  \nonscript\:
  \mathopen{}
  \allowbreak}
#1
}

\DeclarePairedDelimiter{\Bicycle}{(}{)}
% Constants
\newcommand{\CM}{\text{CM}}


\thesisType{Masterarbeit in Informatik}
\date{14. Mai 2016}
% \title{Integration von Unsicherheit in das Reinforcement Learning mit Hilfe von Gaußprozessen}
\title{Incorporating Uncertainty into Reinforcement Learning through Gaussian Processes}
\author{Markus Kaiser}
\matNo{3617190}
\studies{Master Informatik}
\eMail{\href{mailto:markus.kaiser@in.tum.de}{markus.kaiser@in.tum.de}}
\chair{Chair for Foundations of Software Reliability and Theoretical Computer Science}
\university{Fakultät für Informatik \\
    Technische Universität München}
\advisor{Prof. Dr.-Ing. Thomas A. Runkler}
\tutor{Dr. Clemens Otte}

\addbibresource{sources.bib}

\includeonly{chapters/bicycle}

\begin{document}
\frontmatter
\maketitle
% correct BCOR - undo at the end !!!
%\def\bcorcor{0.15cm}
%\addtolength{\hoffset}{\bcorcor}

\cleardoublepage
\thispagestyle{empty}

\makeatletter
\vspace{10mm}
\begin{center}
    %\oTUM{4cm}
    \includegraphics[height=3cm,keepaspectratio]{figures/tum.eps}

    \vspace{5mm}
    \huge DEPARTMENT OF INFORMATICS\\
    \vspace{0.5cm}
    \Large TECHNISCHE UNIVERSITÄT MÜNCHEN\\
\end{center}

%\vspace{10mm}
\begin{center}
    {\Large \thesisTypeText}

    \vspace{10mm}
    {\LARGE Incorporating Uncertainty into Reinforcement Learning through Gaussian Processes}\\
    \vspace{10mm}
    {\Large Integration von Unsicherheit in das Reinforcement Learning mit Hilfe von Gaußprozessen}\\

    \vfill

    %\hfill
    \begin{tabular}{ll}
        \Large Author:     & \Large \@author \\[2mm]
        \Large Supervisor:    & \Large \advisorText \\[2mm]
        \Large Advisors:  & \Large \tutorText\\[2mm]
        \Large  & \Large Prof. Dr. Carl Henrik Ek\\[2mm]
        \Large Date:       & \Large \@date
    \end{tabular}

    \vspace{5mm}

    \includegraphics[height=2cm]{figures/tum_info.eps}
\end{center}
\makeatother

% undo BCOR correction
%\addtolength{\hoffset}{\bcorcor}

\chapter*{Erklärung}
Ich versichere, dass ich diese Masterarbeit selbständig verfasst und nur die angegebenen Quellen und Hilfsmittel verwendet habe.

\vspace{4\baselineskip}

\makeatletter
München, den \@date
\makeatother

\begin{Abstract}{english}
Reinforcement learning is a branch of machine learning concerned with decision-making in an unknown environment in order to achieve high-level goals.
Instead of relying on expert knowledge, an agent has to gain experience through interaction and receives feedback via a numerical reward.
In model-based reinforcement learning, this experience is represented as a model of the system's dynamics which can be used to simulate its future behaviour.
This thesis is concerned with reducing the model bias introduced by choosing actions which are optimal with respect to imperfect models.
Instead of relying on a single deterministic model, gathered knowledge is represented using Gaussian processes which encode a probability distribution over all plausible transition models.
By averaging over all these models, the expected long-term reward is calculated, which explicitly incorporates model uncertainties into long-term planning.
A controller is formulated by applying particle swarm optimization to this expected reward, directly choosing appropriate actions.
Besides formally introducing these tools, this thesis investigates their effectiveness on a benchmark problem with the task of learning how to balance and navigate a bicycle.
Thereby, multiple approaches of incorporating uncertainties are described and compared to the classic technique of deterministic predictions.
\end{Abstract}

\begin{Abstract}{ngerman}
\todo{abstract placeholder}
Diese Arbeit beschäftigt sich mit der Konstruktion von optimierten Flugpfaden für Kamerafahrten mit Multikoptern.
Im Theorieteil werden geeignete Metriken definiert, mittels derer in einem Optimierungsverfahren Splines als Trajektorien erzeugt werden.
Diese Lösung wird in eine Software überführt und um eine Flugsimulation erweitert.
Schließlich werden Schnittstellen zu Flugsystemen der Firma Asctec geschaffen und die Flugbahnen in einem Anwendungsfall getestet.
\end{Abstract}

\tableofcontents

\mainmatter

\chapter{Introduction}

\chapter{The Bicycle Benchmark}
\label{cha:the_bicycle_benchmark}
An example of a system which a computer should learn to control is the bicycle benchmark originally defined by \citeauthor{randlov_learning_1998} in 1998~\cite{randlov_learning_1998}.
The task introduced in this benchmark is to balance a simulated bicycle which travels at a constant speed.

The computer, or agent, takes the place of the rider of the bicycle.
After fixed time intervals, the controller has to decide how to influence the bicycle by applying some torque $T$ on the handle bars to steer, by displacing the center of mass of the bicycle via leaning over the bicycle by some distance $d$ or both.
Besides the short-term goal of preventing the bicycle from falling over, there is also the long-term one of navigating to some predefined goal position.
To make its decision, the controller is given perfect information about the internal state of the simulation.

The bicycle is modelled by non-linear differential equations which describe the steering behaviour and the leaning behaviour of the bicycle.
The speed of the back tyre is considered constant and independent of the actions of the agent.
This results in two important angles to describe the system.
The angle $\theta$ is measured between the front tyre and the frame of the bicycle and describes the straightness of the bicycles path.
The angle $\omega$ measures the amount of tilt of the bicycle's frame from standing upright.
If the absolute value of $\omega$ is greater than \ang{12}, the bicycle has fallen over.
In addition to the two angles, their time-derivatives $\dot{\theta}$ and $\dot{\omega}$ define the current state of the dynamics of the bicycle.
Together with the bicycle's position and rotation in euclidean space, this completely describes the current state of one instance of the bicycle benchmark and is summarized in \cref{tab:bicycle_variables}.

The conservations of angular momentum of the tyres results in interactions between $\theta$ and $\omega$ and their derivatives.
The equations presented in the following describe the dynamics of the system as introduced in~\cite{randlov_learning_1998}.
Two simplifying assumptions have been made:
Firstly, the front fork is assumed to be vertical, which makes balancing the bicycle more difficult but is not physically impossible.
And secondly, the equations are not an exact analytical descriptions, as some second and higher order terms have been ignored.
\begin{table}[p]
    \centering
    \caption{Variables which define the current state in the bicycle system.}
    \label{tab:bicycle_variables}
    \begin{tabularx}{\tablewidth}{cXcc}
        \toprule
        Notation & Description & Value range & Unit \\
        \midrule
        $\theta$ & Angle of the handlebar and the front tyre & \numrange[parse-numbers=false]{-\sfrac{\pi}{2}}{\sfrac{\pi}{2}} & \si{\radian} \\
        $\dot{\theta}$ & Rotational speed of the handlebars & \numrange{-10}{10} & \si{\radian\per\second} \\
        $\omega$ & Tilt of the bike & \numrange[parse-numbers=false]{-\sfrac{\pi}{15}}{\sfrac{\pi}{15}} & \si{\radian} \\
        $\dot{\omega}$ & Tilting speed of the bike & \numrange{-10}{10} & \si{\radian\per\second} \\
        $x$ & Global $x$-position of front tyre & \numrange{-100}{100} & \si{\metre} \\
        $y$ & Global $y$-position of front tyre & \numrange{-100}{100} & \si{\metre} \\
        $\psi$ & Global orientation of the bike & \numrange[parse-numbers=false]{-\pi}{\pi} & \si{\radian} \\
        \bottomrule
    \end{tabularx}
\end{table}
\begin{table}[p]
    \centering
    \caption{Actions which can be applied to the bicycle system.}
    \label{tab:bicycle_actions}
    \begin{tabularx}{\tablewidth}{cXcc}
        \toprule
        Notation & Description & Value range & Unit \\
        \midrule
        $d$ & The distance the agent leans sideways by displacing the center of mass & \numrange{-0.02}{0.02} & \si{\metre} \\
        $T$ & The torque the agent applies on handlebars & \numrange{-2}{2} & \si{\N} \\
        \bottomrule
    \end{tabularx}
\end{table}
\begin{table}[p]
    \centering
    \caption{Physical constants and their values in the bicycle system~\cite{randlov_learning_1998}.}
    \label{tab:bicycle_constants}
    \begin{tabularx}{\tablewidth}{cXr}
        \toprule
        Notation & Description & Value \\
        \midrule
        CM & Center of mass of the bicycle and cyclist in total & \\
        $c$ & Horizontal distance between the point where the front tyre touches the ground and the CM & \SI{66}{\cm} \\
        $d_{\CM}$ & Vertical distance between the CM of the bicycle and the cyclist & \SI{30}{\cm} \\
        $h$ & Vertical distance between the CM and the ground & \SI{94}{\cm} \\
        $l$ & Distance between the points where the front and back tyres touch the ground & \SI{111}{\cm} \\
        $M_c$ & Mass of the bicycle & \SI{15}{\kg} \\
        $M_d$ & Mass of a tyre & \SI{1.7}{\kg} \\
        $M_p$ & Mass of the cyclist & \SI{60}{\kg} \\
        $r$ & Radius of a tyre & \SI{34}{\cm} \\
        $v$ & Velocity of the bicycle & \SI{10}{\km\per\hour} \\
        $\dot{\sigma}$ & The angular velocity of a tyre & $\dot{\sigma} = \sfrac{v}{r}$ \\
        \bottomrule
    \end{tabularx}
\end{table}

\begin{figure}[p]
    \centering
    \begin{subfigure}{.9\textwidth}
        \missingfigure[figheight=.5\textheight]{Bicycle as seen from above}
        \caption{Bicycle as seen from above}
        \label{fig:bicycle:above}
    \end{subfigure}
    \begin{subfigure}{.45\textwidth}
        \missingfigure[figheight=.35\textheight]{Bicycle as seen from behind}
        \caption{Bicycle as seen from behind.}
        \label{fig:bicycle:behind}
    \end{subfigure}
    \begin{subfigure}{.45\textwidth}
        \missingfigure[figheight=.35\textheight]{Moments of inertia}
        \caption{Moments of inertia.}
        \label{fig:bicycle:inertia}
    \end{subfigure}
    \caption{Bicycle dynamics}
    \label{fig:bicycle}
\end{figure}

An overview of the geometric interpretations of the state variables can be seen in \cref{fig:bicycle}.
The interactions between tilt and lean are based on the conservation of angular momentum which is heavily influenced by the moments of inertia of the system.
The moments of the tyre as displayed in \cref{fig:bicycle:inertia} and the moment of the bicycle and cyclist combined $I_{\text{BC}}$ were estimated by the original definition~\cite{randlov_learning_1998} as
\begin{align}
    I_{dc} &\coloneqq \hphantom{\frac{1}{2}}M_d r^2\\
    I_{dv} &\coloneqq \frac{3}{2}M_d r^2\\
    I_{dl} &\coloneqq \frac{1}{2}M_d r^2\\
    I_{\text{BC}} &\coloneqq \frac{13}{3} M_c h^2 + M_p(h + d_\CM)^2.
\end{align}
The dynamics also depend on multiple constants which are detailed in \cref{tab:bicycle_constants}.

The angular acceleration $\ddot{\omega}$ of the lean of the bicycle consists of three terms.
Firstly, the gravitation acting on the bicycle and cyclist which pulls the bicycle in the direction it is already leaning.
Secondly, there are effects based on the conservation of angular momentum introduced via a cross-term dependent on \todo{interpretation?} $\dot{\theta}$ and lastly the centrifugal force applied because of the curved movement of the bicycle.
The center of mass can be displaced horizontally by the agent via the choice of $d$.
The combination $\varphi$ of this displacement and the lean angle of the bicycle is defined as
\begin{align}
    \varphi &\coloneqq \omega + \arctan \left( \frac{d}{h} \right).
\end{align}
With this, the angular acceleration $\ddot{\omega}$ can be calculated as
\begin{align}
    \ddot{\omega} \coloneqq \frac{1}{I_{\text{BC}}} \left(
        \sin \varphi \cdot Mgh - \cos \varphi \cdot \left(
            I_{dc}\dot{\sigma}\cdot\dot{\theta} +
            \sgn(\theta) \cdot v^2 \left(
                \frac{M_dr}{r_f} + \frac{M_d r}{r_b} + \frac{Mh}{r_\CM}
            \right)
        \right)
    \right).
\end{align}

The angular acceleration $\ddot{\theta}$ of the orientation of the front tyre and the handlebars is equal by definition.
It is dependent on the torque $T$ applied to the handlebars and the conservation of angular momentum introduced via a cross-term dependent on $\dot{\omega}$ and is defined as
\begin{align}
    \ddot{\theta} \coloneqq \frac{T - I_{dv}\dot{\sigma}\cdot\dot{\omega}}{I_{dl}}.
\end{align}

These differential equations describe the internal dynamics of the bicycle benchmark, namely how leaning and turning the handlebars interact with each other.
What remains is the movement of the bicycle in space.
The bicycle state contains three variables which locate the bicycle in space.
The position of the bicycle is defined by the point where the front tyre touches the ground.
This point is independent of the orientation of the handlebars (given by the angle $\theta$) and identified via the two euclidean coordinates $x$ and $y$.
The last state variable, $\psi$, describes the orientation of the bicycle frame relative to this point as shown in \cref{fig:bicycle:above}.
If $\psi$ is equal to zero, the back tyre points in to $x$-direction if viewed from the point $(x, y)$ and a positive $\psi$ denotes a counter-clockwise rotation\todo{$\psi$ rotation direction correct?}.

The back tyre of the bicycle moves at the constant speed $v$.
The front and back tyre follow two circular paths with different radii but the same center as can be seen in \cref{fig:bicycle:above} with the front tyre following the longer path.
The radii $r_f$ and $r_b$ can be calculated as
\begin{align}
    r_f &= \frac{l}{\abs*{\sin \theta}} \\
    r_b &= \frac{l}{\abs*{\tan \theta}}
\end{align}
respectively for $\theta$ not equal to zero.
The singularity of $\theta$ approaching zero yields radii of arbitrary size as the bicycle's path becomes more and more straight.
If $\theta$ is equal to zero, the bicycle's orientation does not change and it moves along a straight line.
This case will be ignored below.

Since the two tyres are connected with a rigid frame and the front tyre travels on a longer path, it has to do so at a higher speed.
They do share the same angular velocity $v_\circ$ on their respective circular paths however, since if $\theta$ was constant, the bicycle would rotate around the common center point.
This angular velocity can be derived from the constant speed of the back tyre and is given by
\begin{align}
    v_\circ = \frac{v}{r_b}.
\end{align}
Combined with the direction of rotation on the circle defined by the sign of $\theta$, this directly yields the derivative of the world orientation $\psi$:
\begin{align}
    \dot{\psi} = \sgn(\theta) \cdot v_\circ
\end{align}

The actual speed of the front tyre can be obtained from the common angular velocity $v_\circ$ and the radius of its path $r_f$.
Together with the orientation of the front tyre, this gives the derivatives
\begin{align}
    \dot{x} &= v_\circ \cdot r_f \cdot \cos(\psi + \theta) \\
    \dot{y} &= v_\circ \cdot r_f \cdot \sin(\psi + \theta)
\end{align}
of the position of the bicycle.

The original implementation of \citeauthor{randlov_learning_1998} uses an explicit Euler scheme to evolve the dynamics of the system for a time step.
The changes of position and orientation are calculated using the exact analytical solution of moving the tyres along their circular paths.
To improve accuracy, the implementation developed for this thesis implements a standard Runge-Kutta-Scheme as described in Numerical Recipies~\cite[908]{press_numerical_2007}.
For ease of implementation, the exact solutions for the orientation and position are also evaluate using Runge-Kutta.

The goal of the bicycle benchmark is to drive the bicycle to the goal area which is assumed to be a circle at the origin of the coordinate system with a radius of 5.
The bicycle starts in a position which is almost upright, with the state variables $\theta$, $\dot{\theta}$, $\omega$, $\dot{\omega}$ being sampled with from Gaussian noise with a standard deviation of one percent of their value range.

The agent has to choose the two actions $d$ and $T$ (see \cref{tab:bicycle_actions}) every 0.01 seconds, after which they are assumed constant for this time interval.
After this time, the controller is presented with the current and exact values of the state variables and has to make a new decision.
The bicycle system is therefore a benchmark which creates time series data.
Every such time series or episode starts with some predefined start state and consists of pairs of states and the actions chosen by the agent for this state.
The episode ends in failure if either the bicycle falls over and it ends in success if the bicycle reaches the goal.
Since it is possible for the bicycle to drive in circles indefinitely, the episode also ends in failure after a certain amount of time has passed.

The next chapter will introduce the current state of the art of solving problems such as the bicycle benchmark using model based reinforcement learning.
These standard methods will then be applied to the benchmark, followed by approaches incorporating information about uncertainties.


\chapter{State of the Art}
\section{Reinforcement Learning}
\subsection{Problem statement}
\section{Dynamics: Gaussian Processes}
\subsection{Definition}
\subsection{Kernel functions}
\subsection{Regression with GPs}
\subsection{Sparse Approximations (SPGP)}
\section{Particle Swarm Optimization}
\subsection{Basic PSO}
\subsection{Improvements}

\chapter{Solution of the Bicycle Benchmark}
\section{Standard Approach: Means Only}
\subsection{Dynamics: with GP Models}
\subsection{Results and Discussion of Problems}
\subsection{New Reward Function Reward function}
\section{Our second Approach: Linearization}

\chapter{Conclusion}
\section{Discussion of the Approaches}
\section{Possible Improvements}

\appendix

\nocite{*}
\printbibliography
\listoffigures
\listoftables
\listoflistings

\end{document}
